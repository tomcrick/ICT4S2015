\documentclass[conference]{IEEEtran}

\usepackage[british]{babel}
\usepackage{cite}
\usepackage{graphicx}
\usepackage[hyphens]{url}
%\usepackage[pdftex]{hyperref}


% correct bad hyphenation here
%\hyphenation{op-tical net-works semi-conduc-tor}


\begin{document}
%
% paper title
% can use linebreaks \\ within to get better formatting as desired
\title{The Environmental Impacts of IT Use:\\A User-Oriented Perspective}


% author names and affiliations
% use a multiple column layout for up to three different
% affiliations
\author{\IEEEauthorblockN{Peter Cooper}
\IEEEauthorblockA{Faculty of Engineering\\
University of Bristol\\
Bristol, UK\\
Email: peter.cooper@bristol.ac.uk}
\and
\IEEEauthorblockN{Tom Crick}
\IEEEauthorblockA{Department of Computing\\
Cardiff Metropolitan University\\
Cardiff, UK\\
Email: tcrick@cardiffmet.ac.uk}
\and
\IEEEauthorblockN{Theo Tryfonas}
\IEEEauthorblockA{Faculty of Engineering\\
University of Bristol\\
Bristol, UK\\
Email: theo.tryfonas@bristol.ac.uk}}

% conference papers do not typically use \thanks and this command
% is locked out in conference mode. If really needed, such as for
% the acknowledgment of grants, issue a \IEEEoverridecommandlockouts
% after \documentclass


% use for special paper notices
%\IEEEspecialpapernotice{(Invited Paper)}


% make the title area
\maketitle


\begin{abstract}
This paper discusses the environmental impact of the use of
Information Technology, with a special focus on how individual user
behaviour affects this impact. By using a process life cycle
assessment, we are able to estimate the environmental impact of the IT
use of an individual within the UK over a one year period.

By estimating the energy and data consumption of an average user's
average use of an average device, and estimating the associated energy
usage (and thus CO2 produced) of each stage in the data chain, summed
to a CO2 value for embodied carbon of an average device. To enable
this, market segmentation is undertaken to determine the average user
and average use, along with market research to determine the average
device. Analysis of device performance in difference behaviours is
also undertaken.

Overall, device energy is seen to dominate; within device, desktops
dominate, both due to their high energy use for a given task, but also
their high standby power, which is the most significant point of
behaviour-driven waste. Geographical, behavioural and chronological
factors are all evaluated to be highly significant to the impact of a
user's IT use, along with a number of secondary factors discovered in
the evaluation.

Finally, we review the current domain research, to advise the
secondary aims of furthering the understanding of the factors
affecting the environmental impact of IT and to recommend practical
improvements based on newly assessed factors.
\end{abstract}

% For peer review papers, you can put extra information on the cover
% page as needed:
% \ifCLASSOPTIONpeerreview
% \begin{center} \bfseries Smart Cities, Public Value Management, Leadership, information marketplaces \end{center}
% \fi
%
% For peerreview papers, this IEEEtran command inserts a page break and
% creates the second title. It will be ignored for other modes.
%\IEEEpeerreviewmaketitle

% \begin{IEEEkeywords}
% Information Communications Technology, Energy, Greenhouse Gas (GHG)
% Emissions, Life Cycle Assessment, Chronological Variation, Use Case, GUI
% \end{IEEEkeywords}



%\newpage

% trigger a \newpage just before the given reference
% number - used to balance the columns on the last page
% adjust value as needed - may need to be readjusted if
% the document is modified later
%\IEEEtriggeratref{28}
% The "triggered" command can be changed if desired:
%\IEEEtriggercmd{\enlargethispage{-5in}}

% references section
\bibliographystyle{IEEEtran}
\bibliography{ict4s2015}

% that's all folks
\end{document}


